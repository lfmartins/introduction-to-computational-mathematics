\documentclass[12pt]{article}
\RequirePackage{amsmath}
\RequirePackage{amsfonts}
\RequirePackage{amssymb}
\RequirePackage{amsthm}
\RequirePackage{mathtools}
\RequirePackage{graphicx}
\RequirePackage{mathabx}
\RequirePackage{varioref}
\RequirePackage{cancel}
\RequirePackage{MnSymbol,wasysym}
\RequirePackage{fancybox}
\RequirePackage{pdfpages}
\RequirePackage{systeme}
\RequirePackage{mathrsfs}
\RequirePackage{tabularx}

\RequirePackage{listings}
\lstset{language=Python, 
   basicstyle=\ttfamily\small, 
   showstringspaces=false,
   tabsize=4,
   frameround=tttt,
   captionpos=b}

\lstnewenvironment{lstfancy}[2]
{\lstset{float=th,caption=#1,label=#2,numbers=left,numberstyle=\tiny,%
frame=tb,escapeinside={(*}{*)}}}
{}

\RequirePackage[colorlinks=true,
	linkcolor=blue, urlcolor=blue,
	pdfauthor={L. Felipe Martins},
	pdfsubject={Computational number theory},
	pdfkeywords={number theory, computations, cryptography},
	pdfstartview=FitH,
	pdfview=FitH]{hyperref}

\RequirePackage{fancyhdr}

\RequirePackage{titlesec}
\titleformat{\section}{\large\bf\sffamily}{\thesection}{1em}{}
\titleformat{\subsection}{\bf\sffamily}{\thesubsection}{1em}{}

\newenvironment{remarkbox}%
{\begin{Sbox}\begin{minipage}{\linewidth}}%
{\end{minipage}\end{Sbox}\begin{center}\shadowbox{\TheSbox}\end{center}}%


%----------------------------------------------------------
%\swapnumbers
\theoremstyle{plain}
\newtheorem{theorem}{Theorem}[section]
\newtheorem{proposition}[theorem]{Proposition}
\newtheorem{lemma}[theorem]{Lemma}
\newtheorem{corollary}[theorem]{Corollary}
\newtheorem{conjecture}[theorem]{Conjecture}
\newtheorem{criterion}[theorem]{Criterion}
\newtheorem{principle}[theorem]{Principle}
\newtheorem{property}[theorem]{Property}
\newtheorem{claim}[theorem]{Claim}

\theoremstyle{definition}
\newtheorem{definition}[theorem]{Definition}
\newtheorem{condition}[theorem]{Condition}
\newtheorem{example}[theorem]{Example}
\newtheorem{exercise}[theorem]{Exercise}
\newtheorem{solution}[theorem]{Solution}
\newtheorem{problem}[theorem]{Problem}

\theoremstyle{remark}
\newtheorem{remark}[theorem]{Remark}
\newtheorem{note}[theorem]{Note}
\newtheorem{notation}[theorem]{Notation}
%----------------------------------------------------------

%My definitions

%Number sets
\newcommand\Z{\ensuremath{\mathbb{Z}}}
\newcommand\N{\ensuremath{\mathbb{N}}}
\newcommand\R{\ensuremath{\mathbb{R}}}
\newcommand\Q{\ensuremath{\mathbb{Q}}}
\newcommand\C{\ensuremath{\mathbb{C}}}
\newcommand\K{\ensuremath{\mathbb{K}}}

% Floor and ceil functions
\newcommand{\floorf}[1]{\ensuremath{\left\lfloor{#1}\right\rfloor}}
\newcommand{\ceilf}[1]{\ensuremath{\left\lceil{#1}\right\rceil}}

% "Divides" and "Not divides"
\newcommand{\dv}{\ensuremath{\mathbf{\divides}}}
\newcommand{\ndv}{\ensuremath{\notdivides}}

%Shortcut to underline and cancel
\newcommand{\un}[1]{\underline{#1}}
\newcommand{\cn}[1]{\cancel{#1}}

% Sets
\newcommand{\setof}[2]{\ensuremath{\left\{ #1 \; | \; #2 \right\}}}
\newcommand{\setofc}[2]{\ensuremath{\left\{ #1 : #2 \right\}}}
\newcommand{\setenum}[1]{\left\{#1\right\}}

% Prime factorizations
%\newcommand{\primefactorization}[3]{\ensuremath{#1_1^{#2_1}#1_2^{#2_2}\ldots#1_{#3}^{#2_{#3}}}}
\newcommand{\primefactorization}[3]{#1_1^{#2_1}#1_2^{#2_2}\ldots#1_{#3}^{#2_{#3}}}
\newcommand{\primefactorizationprod}[4]{\ensuremath{\prod_{#4=1}^{#3}#1_{#4}^{#2_{#4}}}}

% Modular identities
\newcommand{\eqmod}[3]{\ensuremath{#1 \equiv #2 \pmod{#3}}}

% End of proof
\newcommand{\proofend}{\hfill  {\Large\smiley}}%


%Legendre symbols
%\newcommand{\legendre}[2]{ {#1}\abovewithdelims () {#2}}
\newcommand{\legendre}[2]{\left(\frac{#1}{#2}\right)}

% Operators
%\DeclareMathOperator{\bdiv}{div}
\newcommand{\bdiv}{\;\mathrm{div}\;}
\DeclareMathOperator{\round}{round}
%\DeclareMathOperator{\divmod}{divmod}
%\DeclareMathOperator{\bdivp}{divp}
%\DeclareMathOperator{\bmodp}{modp}
%\DeclareMathOperator{\divmodp}{divmodp}
\DeclareMathOperator{\ord}{ord}
\DeclareMathOperator{\dlog}{dlog}
\DeclareMathOperator{\laspan}{span}
\DeclareMathOperator{\rank}{rank}
\DeclareMathOperator{\re}{Re}
\DeclareMathOperator{\im}{Im}
\DeclareMathOperator{\sgn}{sgn}
\DeclareMathOperator{\Arg}{Arg}
\DeclareMathOperator{\cis}{cis}
\DeclareMathOperator{\dist}{dist}
\DeclareMathOperator{\diam}{diam}
\DeclareMathOperator{\length}{length}

%Equation reference formats
\labelformat{equation}{(#1)}

%Macro to introduce license notice
%\newcommand{\license}%
%{{\scriptsize Work licensed under a Creative Commons License available at}\\%
%{\scriptsize\url{http://creativecommons.org/licenses/by-nc-sa/3.0/us/}}\\}

%--------------------- Formatting for number theory modules ------------------------
% Set headings
\newcommand{\setheadings}[1]{%
\fancyhead[L]{{\large \textsf{#1}}}%
\fancyhead[R]{\thepage}%
\fancyfoot[L]{%
{\scriptsize L. Felipe Martins (\href{mailto:l.martins@csuohio.edu}{l.martins@csuohio.edu}),
Alexander P. Hoover (\href{mailto:a.p.hoover@csuohio.edu}{a.p.hoover@csuohio.edu)}\\%
%Licensed under a Creative Commons License, visit %\url{http://creativecommons.org/licenses/by-nc-sa/3.0/} to view a copy of the license.%
}}}%

\newcommand{\setheadingsa}[2]{%
\fancyhead[L]{{\large \textsf{#1}}}%
\fancyhead[R]{\thepage}%
\fancyfoot[L]{\scriptsize{#2}}%
}%

\renewcommand{\footrulewidth}{0.4pt}%

% Change margins
\setlength{\headheight}{29pt}
\pagestyle{fancy}
\fancyhead{}
\fancyfoot{}
\usepackage[left=2cm,right=2cm,bottom=2cm]{geometry}
\fancyheadoffset[L]{0cm}

\numberwithin{equation}{section}

\setlength{\parindent}{0pt}
\setlength{\parskip}{0.5em}

% Macros for vectors
\newcommand{\bv}[1]{\ensuremath\mathbf{#1}}
\newcommand{\vcp}[2]{\left\langle #1,#2 \right\rangle}
\newcommand{\vcs}[3]{\left\langle #1,#2,#3 \right\rangle}
\newcommand{\vcn}[3]{\left\langle #1,#2,\dots,#3 \right\rangle}
\newcommand{\vu}{\bv{u}}
\newcommand{\vv}{\bv{v}}
\newcommand{\vw}{\bv{w}}
\newcommand{\va}{\bv{a}}
\newcommand{\vb}{\bv{b}}
\newcommand{\vc}{\bv{c}}
\newcommand{\vd}{\bv{d}}
\newcommand{\ve}{\bv{e}}
\newcommand{\vf}{\bv{f}}
\newcommand{\vh}{\bv{h}}
\newcommand{\vi}{\bv{i}}
\newcommand{\vj}{\bv{j}}
\newcommand{\vr}{\bv{r}}
\newcommand{\vs}{\bv{s}}
\newcommand{\vk}{\bv{k}}
\newcommand{\vx}{\bv{x}}
\newcommand{\vy}{\bv{y}}
\newcommand{\vn}{\bv{n}}
\newcommand{\vF}{\bv{F}}
\newcommand{\vS}{\bv{S}}
\newcommand{\vT}{\bv{T}}
\newcommand{\vA}{\bv{A}}
\newcommand{\vB}{\bv{B}}
\newcommand{\vE}{\bv{E}}

% Macros for logic
\newcommand{\limp}{\rightarrow}
\newcommand{\leqv}{\leftrightarrow}
\newcommand{\lxor}{\oplus}
\newcommand{\lnand}{\uparrow}
\newcommand{\lnor}{\downarrow}

% Macro for including a centered image. 
% First argument is scaling, second argument is image file
% Assumes image file is in subdirectory "images" of the current directory.
\newcommand{\centeredimage}[2]{%
\begin{center}%
\includegraphics[scale=#1]{images/#2}%
\end{center}%
}




\setheadings{MTH 593 --- Introduction to Computational Mathematics}

\begin{document}

\sffamily

\begin{center}


\bigskip
{\large \textbf{Syllabus for MTH 593 --- Introduction to Computational Mathematics}}

Fall 2022: August 29 -- December 17
\end{center}

\section{Course Information}
\begin{tabular}{ll}
\textbf{Instructor}: & Dr. L. Felipe Martins.\\
\textbf{Office}: &  RT1533\\
\textbf{E-mail}: &  \href{mailto:l.martins@csuohio.edu}{l.martins@csuohio.edu}\\
%\textbf{Phone}: & 216--687--4683 \\
\textbf{Office Hours}: & MWF 2:30pm--3:30pm or by appointment\\
%\textbf{}: &  \\
\end{tabular}


\section{Recommended Textbook}

Lutz, Mark \emph{Learning Python}, 5th ed (June 2013).

The textbook is available online free of charge to CSU students. Go the the library e-books page:
\begin{center}
\url{https://library.csuohio.edu/research/vrd/ebooks.html}
\end{center}
and search for \emph{Safari Books Online}. Click on the link and create an account. Once you access the Safari site, search for \emph{Learning Python}.


\section{Course Description and Topics}

\bigskip
This course is an introduction to computational mathematics using the Python computer language.
\begin{enumerate}
\item Introduction to Python
\begin{enumerate}
\item Programming Python in the Jupyter Notebook.
\item Operations with integers and doubles, including built-in mathematical functions.
\item Using variables to store values.
\item Using lists to store collections of values.
\item Using strings to represent textual information.
\item Programming structures: if tests, for loops and while loops
\item Functions.
\item Introduction to object oriented programming from the point of view of using libraries. 
\item Introduction to the Python documentation.
\end{enumerate}
\item Modules, packages and the standard library.
\item Numpy: efficient arrays for scientific computing.
\item Mathplotlib: professional quality graphs
\item Scipy: advanced scientific computing algorithms.
\item Pandas: structures for data science
\item Scikit-learn: machine learning algorithms (time permitting).

\end{enumerate}

\section{Learning Resources}
\begin{itemize}
\item \textbf{Blackboard Learn (BBLearn)}: This course uses the CSU online course 
management system: 
\href{https://www.csuohio.edu/center-for-elearning/blackboard-login}{https://www.csuohio.edu/center-for-elearning/blackboard-login}. 
Visit the site frequently for course information, discussion boards, supplemental material, useful 
links, and other resources.
\end{itemize}


\section{Course work and grading}

\subsection{Class Participation (10 points)} 
This is a hands-on course and students are expected to actively participate in class. Class meetings will be structured around learning activities, and working on these activities is essential to be successful in the course. Participation credit is given according to completion of in-class tasks. At the end of the semester, points are awarded for participation according to the following table:

\begin{center}
\begin{tabular}{|c|c|}\hline
\textbf{Participation score} & \textbf{Participation points}\\\hline\hline
90\% or more & 10\\\hline
80\% to 90\% & 5\\\hline
less than 80\% & 0\\\hline
\end{tabular}
\end{center}

\subsection{Quizzes (20 points)}
Quizzes are designed to check the understanding of basic programming concepts. Each quiz is a short in-class test that requires the use of a computer. The topics and dates of quizzes will be announced at least three days before the quiz takes place.

When computing the quiz score at the end of the semester, the lowest score is dropped. Please notice that \emph{no makeups are given for missed quizzes}. Students that, due to a health or family emergency, are required to have an extended absence should contact the instructor immediately to make arrangements to make up for missed work.


\subsection{Projects (70 points)}

Students will be required to work on several projects during the course, including a final project. Projects explore applications of the programming concepts studied in class. The final project is designed to give students an opportunity to work on a major application area. Project topics will be announced during the course.




\medskip
\textbf{Calculator Policy}: 


\subsection{Grade assignment} Grades are assigned according to the following table:

\begin{center}
  \begin{tabular}[t]{c|cl}
    \multicolumn{3}{c}{\textbf{Undergraduate}}\\
    \hline
    \textbf{Score} & \multicolumn{2}{c}{\textbf{Grade}}\\
    \hline
    95 to 100	&& \ \texttt{A}\\
    90 to 94 	&& \ \texttt{A-}\\
    85 to 89 	&& \ \texttt{B+}\\
    83 to 84	&& \ \texttt{B}  \\
    80 to 82	&& \ \texttt{B-} \\
    75 to 79	&& \ \texttt{C+} \\
    70 to 74	&& \ \texttt{C}  \\
    60 to 69	&& \ \texttt{D}  \\
    59 or less  && \ \texttt{F}
  \end{tabular}
\end{center}

\section{General Policies}

\subsection {Grade reporting and disputes} All student scores will be posted in Blackboard
as course work and tests are done. Students are responsible for checking their own progress, and
reporting to the instructor any discrepancies as soon as they are noticed. It is also strongly suggested
that you retain all graded work from the course until the end of the semester and grades are posted.
This way if a dispute arises concerning a recorded grade and the actual grade, we have the documentation
needed to rectify the situation. Additionally, graded works make for excellent study materials for upcoming
exams.  

\subsection{Class Conduct}
Class attendance and participation is essential for success in this course. Please come to class prepared,
and take an active role in class discussions and activities.

Please bring a graphing calculator to each class.  Cell phones should be turned off or placed on silent.
Text messaging during class is not appropriate and grounds for removal from class.  During computer lab 
sessions, checking email and surfing the web is inappropriate when the instructor is talking and again 
grounds for removal from the class.  Other serious disruptions are grounds for removal as well.

\subsection{Withdrawals}
Last day to withdraw is \textbf{November 4}. Withdrawing from the course may put you in violation
of the federally mandated standards for academic progress (SAP) that you must maintain to be eligible for 
financial aid.  Read the link on the course website for information about the implications of withdrawing 
from the course for your financial aid or visit Campus 411.

\subsection{Scholastic Dishonesty}
Cheating and/or plagiarism will not be tolerated. ``Cheating'' includes copying or receiving help from
another student on quizzes, tests or exams, as well as allowing another student to copy from your work. 
Copying another student's homework, or allowing someone else to do your homework for you, is also 
considered cheating. If cheating occurs in a quiz or unit test, the student will receive a grade of 0 for 
that component of the course. If cheating occurs in the final exam, the student will receive a grade of F in 
the course. Any cheating activity may be reported for further action.  Information regarding the official
CSU policy regarding cheating and plagiarism can be found in the  CSU Code of Student Conduct at 
\url{https://www.csuohio.edu/compliance/student-code-conduct}

\subsection{Disabilities Statement} 
Educational access is the provision of classroom accommodations, auxiliary aids and services to ensure 
equal educational opportunities for all students regardless of their disability. Any student who feels 
he or she may need an accommodation based on the impact of a disability should contact the Office of 
Disability Services at (216) 687-2015. The Office is located in MC 147. Accommodations need to be 
requested in advance and will not be granted retroactively.

\subsection{Disclaimer}
The course instructor reserves the right to modify these procedures as the course progresses, and to change the assignment schedule from the outline given. Any changes will 
be announced in class with adequate advance notice. You are responsible for being aware of any changes 
discussed in class and/or in the BBLearn course site. This includes exam days, homework due dates and changes 
in policy.

%\includepdf[pages={-}]{schedule-mth281-summer2022.pdf}

\end{document}
