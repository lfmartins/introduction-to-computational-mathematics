\documentclass[12pt]{article}
\RequirePackage{amsmath}
\RequirePackage{amsfonts}
\RequirePackage{amssymb}
\RequirePackage{amsthm}
\RequirePackage{mathtools}
\RequirePackage{graphicx}
\RequirePackage{mathabx}
\RequirePackage{varioref}
\RequirePackage{cancel}
\RequirePackage{MnSymbol,wasysym}
\RequirePackage{fancybox}
\RequirePackage{pdfpages}
\RequirePackage{systeme}
\RequirePackage{mathrsfs}
\RequirePackage{tabularx}

\RequirePackage{listings}
\lstset{language=Python, 
   basicstyle=\ttfamily\small, 
   showstringspaces=false,
   tabsize=4,
   frameround=tttt,
   captionpos=b}

\lstnewenvironment{lstfancy}[2]
{\lstset{float=th,caption=#1,label=#2,numbers=left,numberstyle=\tiny,%
frame=tb,escapeinside={(*}{*)}}}
{}

\RequirePackage[colorlinks=true,
	linkcolor=blue, urlcolor=blue,
	pdfauthor={L. Felipe Martins},
	pdfsubject={Computational number theory},
	pdfkeywords={number theory, computations, cryptography},
	pdfstartview=FitH,
	pdfview=FitH]{hyperref}

\RequirePackage{fancyhdr}

\RequirePackage{titlesec}
\titleformat{\section}{\large\bf\sffamily}{\thesection}{1em}{}
\titleformat{\subsection}{\bf\sffamily}{\thesubsection}{1em}{}

\newenvironment{remarkbox}%
{\begin{Sbox}\begin{minipage}{\linewidth}}%
{\end{minipage}\end{Sbox}\begin{center}\shadowbox{\TheSbox}\end{center}}%


%----------------------------------------------------------
%\swapnumbers
\theoremstyle{plain}
\newtheorem{theorem}{Theorem}[section]
\newtheorem{proposition}[theorem]{Proposition}
\newtheorem{lemma}[theorem]{Lemma}
\newtheorem{corollary}[theorem]{Corollary}
\newtheorem{conjecture}[theorem]{Conjecture}
\newtheorem{criterion}[theorem]{Criterion}
\newtheorem{principle}[theorem]{Principle}
\newtheorem{property}[theorem]{Property}
\newtheorem{claim}[theorem]{Claim}

\theoremstyle{definition}
\newtheorem{definition}[theorem]{Definition}
\newtheorem{condition}[theorem]{Condition}
\newtheorem{example}[theorem]{Example}
\newtheorem{exercise}[theorem]{Exercise}
\newtheorem{solution}[theorem]{Solution}
\newtheorem{problem}[theorem]{Problem}

\theoremstyle{remark}
\newtheorem{remark}[theorem]{Remark}
\newtheorem{note}[theorem]{Note}
\newtheorem{notation}[theorem]{Notation}
%----------------------------------------------------------

%My definitions

%Number sets
\newcommand\Z{\ensuremath{\mathbb{Z}}}
\newcommand\N{\ensuremath{\mathbb{N}}}
\newcommand\R{\ensuremath{\mathbb{R}}}
\newcommand\Q{\ensuremath{\mathbb{Q}}}
\newcommand\C{\ensuremath{\mathbb{C}}}
\newcommand\K{\ensuremath{\mathbb{K}}}

% Floor and ceil functions
\newcommand{\floorf}[1]{\ensuremath{\left\lfloor{#1}\right\rfloor}}
\newcommand{\ceilf}[1]{\ensuremath{\left\lceil{#1}\right\rceil}}

% "Divides" and "Not divides"
\newcommand{\dv}{\ensuremath{\mathbf{\divides}}}
\newcommand{\ndv}{\ensuremath{\notdivides}}

%Shortcut to underline and cancel
\newcommand{\un}[1]{\underline{#1}}
\newcommand{\cn}[1]{\cancel{#1}}

% Sets
\newcommand{\setof}[2]{\ensuremath{\left\{ #1 \; | \; #2 \right\}}}
\newcommand{\setofc}[2]{\ensuremath{\left\{ #1 : #2 \right\}}}
\newcommand{\setenum}[1]{\left\{#1\right\}}

% Prime factorizations
%\newcommand{\primefactorization}[3]{\ensuremath{#1_1^{#2_1}#1_2^{#2_2}\ldots#1_{#3}^{#2_{#3}}}}
\newcommand{\primefactorization}[3]{#1_1^{#2_1}#1_2^{#2_2}\ldots#1_{#3}^{#2_{#3}}}
\newcommand{\primefactorizationprod}[4]{\ensuremath{\prod_{#4=1}^{#3}#1_{#4}^{#2_{#4}}}}

% Modular identities
\newcommand{\eqmod}[3]{\ensuremath{#1 \equiv #2 \pmod{#3}}}

% End of proof
\newcommand{\proofend}{\hfill  {\Large\smiley}}%


%Legendre symbols
%\newcommand{\legendre}[2]{ {#1}\abovewithdelims () {#2}}
\newcommand{\legendre}[2]{\left(\frac{#1}{#2}\right)}

% Operators
%\DeclareMathOperator{\bdiv}{div}
\newcommand{\bdiv}{\;\mathrm{div}\;}
\DeclareMathOperator{\round}{round}
%\DeclareMathOperator{\divmod}{divmod}
%\DeclareMathOperator{\bdivp}{divp}
%\DeclareMathOperator{\bmodp}{modp}
%\DeclareMathOperator{\divmodp}{divmodp}
\DeclareMathOperator{\ord}{ord}
\DeclareMathOperator{\dlog}{dlog}
\DeclareMathOperator{\laspan}{span}
\DeclareMathOperator{\rank}{rank}
\DeclareMathOperator{\re}{Re}
\DeclareMathOperator{\im}{Im}
\DeclareMathOperator{\sgn}{sgn}
\DeclareMathOperator{\Arg}{Arg}
\DeclareMathOperator{\cis}{cis}
\DeclareMathOperator{\dist}{dist}
\DeclareMathOperator{\diam}{diam}
\DeclareMathOperator{\length}{length}

%Equation reference formats
\labelformat{equation}{(#1)}

%Macro to introduce license notice
%\newcommand{\license}%
%{{\scriptsize Work licensed under a Creative Commons License available at}\\%
%{\scriptsize\url{http://creativecommons.org/licenses/by-nc-sa/3.0/us/}}\\}

%--------------------- Formatting for number theory modules ------------------------
% Set headings
\newcommand{\setheadings}[1]{%
\fancyhead[L]{{\large \textsf{#1}}}%
\fancyhead[R]{\thepage}%
\fancyfoot[L]{%
{\scriptsize L. Felipe Martins (\href{mailto:l.martins@csuohio.edu}{l.martins@csuohio.edu}),
Alexander P. Hoover (\href{mailto:a.p.hoover@csuohio.edu}{a.p.hoover@csuohio.edu)}\\%
%Licensed under a Creative Commons License, visit %\url{http://creativecommons.org/licenses/by-nc-sa/3.0/} to view a copy of the license.%
}}}%

\newcommand{\setheadingsa}[2]{%
\fancyhead[L]{{\large \textsf{#1}}}%
\fancyhead[R]{\thepage}%
\fancyfoot[L]{\scriptsize{#2}}%
}%

\renewcommand{\footrulewidth}{0.4pt}%

% Change margins
\setlength{\headheight}{29pt}
\pagestyle{fancy}
\fancyhead{}
\fancyfoot{}
\usepackage[left=2cm,right=2cm,bottom=2cm]{geometry}
\fancyheadoffset[L]{0cm}

\numberwithin{equation}{section}

\setlength{\parindent}{0pt}
\setlength{\parskip}{0.5em}

% Macros for vectors
\newcommand{\bv}[1]{\ensuremath\mathbf{#1}}
\newcommand{\vcp}[2]{\left\langle #1,#2 \right\rangle}
\newcommand{\vcs}[3]{\left\langle #1,#2,#3 \right\rangle}
\newcommand{\vcn}[3]{\left\langle #1,#2,\dots,#3 \right\rangle}
\newcommand{\vu}{\bv{u}}
\newcommand{\vv}{\bv{v}}
\newcommand{\vw}{\bv{w}}
\newcommand{\va}{\bv{a}}
\newcommand{\vb}{\bv{b}}
\newcommand{\vc}{\bv{c}}
\newcommand{\vd}{\bv{d}}
\newcommand{\ve}{\bv{e}}
\newcommand{\vf}{\bv{f}}
\newcommand{\vh}{\bv{h}}
\newcommand{\vi}{\bv{i}}
\newcommand{\vj}{\bv{j}}
\newcommand{\vr}{\bv{r}}
\newcommand{\vs}{\bv{s}}
\newcommand{\vk}{\bv{k}}
\newcommand{\vx}{\bv{x}}
\newcommand{\vy}{\bv{y}}
\newcommand{\vn}{\bv{n}}
\newcommand{\vF}{\bv{F}}
\newcommand{\vS}{\bv{S}}
\newcommand{\vT}{\bv{T}}
\newcommand{\vA}{\bv{A}}
\newcommand{\vB}{\bv{B}}
\newcommand{\vE}{\bv{E}}

% Macros for logic
\newcommand{\limp}{\rightarrow}
\newcommand{\leqv}{\leftrightarrow}
\newcommand{\lxor}{\oplus}
\newcommand{\lnand}{\uparrow}
\newcommand{\lnor}{\downarrow}

% Macro for including a centered image. 
% First argument is scaling, second argument is image file
% Assumes image file is in subdirectory "images" of the current directory.
\newcommand{\centeredimage}[2]{%
\begin{center}%
\includegraphics[scale=#1]{images/#2}%
\end{center}%
}




\setheadings{MTH 493/593 --- Introduction to Computational Mathematics --- Project 1}

\begin{document}
\sffamily

\section{Submission Instructions}
The project must be submitted through BBLearn as a Jupyter notebook. Create a new notebook in Google Colab. Please name your notebook as follows:

\begin{center}
Project 1 - Your Name - Your CSU ID
\end{center}

For example, Alice Mathperson with ID 1234567 would name her notebook:

\begin{center}
Project 1 - Alice Mathperson - 1234567
\end{center}

You are allowed to make multiple submissions in BBLearn. However, only the final submission will be graded. In your final submission, make sure of the following:

\begin{itemize}
\item The notebook must run without errors. Before submitting the notebook, check that all cells execute correctly, \emph{in the order they appear in the notebook}.
\item There is one exception to the rule above. Sometimes it is useful that we know, or suspect, to produce errors. In this case, provide an explanation in a text cell for the error.
\item Add text cells to explain what you are doing in each step, and state your conclusions. Your project should be readable as a mathematical paper.
\item Your code should not only be correct, but well structured and clear. Use meaningful variable names and add comments where you feel necessary.
\end{itemize}

\section{Project Statement}
Let's suppose that we need to repeatedly solve quadratic equations, for a large number of cases. In this project, you will define a function \texttt{quadratic\_equation} that outputs the solutions to a the quadratic equation $ax^2+bx+c=0$ given the coefficients $a$, $b$ $c$. We start with a reasonably simple implementation, and then add features to handle more complicated cases. At the end, we will have professional quality code to solve quadratic equations.

\subsection{Part 1 --- Define and test your function}
Write a function that implements the quadratic formula:
\[
x_1,x_2=\frac{-b\pm \sqrt{b^2-4ac}}{2a}
\]
Your code will have the following structure:

\begin{lstlisting}
def quadratic_equation(a, b, c):
    ... formula breakdown ...
    x1 = ...
    x2 = ...
    return x1, x2
\end{lstlisting}

It is recommended that long formulas are avoided, by doing the computation in a series of statements. For example, the function can start by computing the discriminant $b^2-4ac$, and then using this partial computation in the evaluation of the solutions.

Test your code with at least ten test cases. Each test case consists of values for the coefficients $a$, $b$, $c$. Make sure to include tests for the following cases:

\begin{itemize}
\item The solutions are real and distinct.
\item The solutions are not real.
\item The two solutions are identical.
\item The coefficients are complex numbers.
\item The coefficient $a$ is zero.
\end{itemize}

Make notes of any errors that occur.

\subsection{Part 2 --- Handling Complex Solutions}

We now want to polish our function in a way that it produces correct and meaningful results in all cases. We want a function that works in any case, and returns the right type of solution, specially regarding the issue of complex coefficients and/or solutions.

Use the following structure for the function definition:

\begin{lstlisting}
def quadratic_equation(a, b, c):
    if type(a) == complex or type(b) == complex or type(c) == complex:
        ... computation in the case of complex coefficients ...
        x1 = ...
        x2 = ...
    else:
        delta = b ** 2 - 4 * a * c
        if delta >= 0:
            ...  real coefficients, but complex roots ...
            x1 = ...
            x2 = ...
        else:
            ... real coefficients, real roots ...
            x1 = ...
            x2 = ...
return x1, x2
\end{lstlisting}

Test the new version of the function with the same test cases you used in Part 1.

\subsection{Part 3 --- Handling Exceptions}

At this point, there is still one case that your function does not handle well: if the coefficient $a$ is zero. To take care of this case, we will make the function signal an error with a meaningful message. Add the following code right after the \lstinline{def quadratic_equation(a, b,c)}:

\begin{lstlisting}
if a == 0:
    raise ValueError("coefficient a cannot be zero")
\end{lstlisting}

Repeat all tests with the new version of the function.
\end{document}
